For the torus\parencite[p. 180f.]{ar}:

Let $K$ be a triangulation of the torus. If we orient all the 2-simplices of $K$ compatibly, take this sum, and compute the boundary of this sum, then  each edge of the triangulation occurs exactly twice in the result, once with each of its two possible orientations. So we have a two-dimensional cycle. It is elementary to check that any other 2-cycle has to be an integer multiple of this one. (For suppose the oriented triangle $(a,b,c)$ occurs in a 2-cycle with coefficient $\lambda$, then $\lambda(b,c)$ automatically appears in its boundary. Now the edge spanned by $b$ and $c$ lies in precisely one other triangle of $K$ whose third vertex we denote by $d$. The only way we can rid ourselves of the above term $\lambda(b,c)$ is to orient this adjacent triangle as $(d,c,b)$ in other words, compatibly with $(a,b,c)$, and include it in our cycle with the same coefficient $\lambda$. Going round the complex in this way, we see we must orient all simplexes compatibly, and give them all the same coefficient.) Since there are no 3-simplices in a triangulation of the torus, there are no bounding cycles, and therefore $H_2(K)$ is isomorphic to $\Z$.

%Just about punctuerd Torus
%If we now change to a triangulation of the punctured torus, there are no 2-cycles, since even if we include all the triangles oriented compatibly as above, when we compute the boundary we are left with those edges which form the boundary of the hole in the torus. So the second homology group is zero.

for the klein bottle:
The second homology group of a triangulation of the Klein bottle is zero. There are no 2-cycles because it is not possible to orient all the 2-simplices of a triangulation compatibly, since the Klein bottle is nonorientable.

Therefore the homology groups for the torus and the Klein bottle are not isomorphic in every dimension. Thus, they can't be homeomorphic.